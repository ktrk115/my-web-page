\documentclass[paper=a4,fontsize=11pt]{scrartcl} % KOMA-article class

\usepackage{hyperref}
\usepackage[english]{babel}
% \usepackage[utf8x]{inputenc}
\usepackage[protrusion=true,expansion=true]{microtype}
\usepackage{amsmath,amsfonts,amsthm}     % Math packages
\usepackage{graphicx}                    % Enable pdflatex
\usepackage[svgnames]{xcolor}            % Colors by their 'svgnames'
\usepackage{geometry}
    \textheight=700px                    % Saving trees ;-)
\usepackage{url}
\usepackage{enumitem}

\frenchspacing              % Better looking spacings after periods
\pagestyle{empty}           % No pagenumbers/headers/footers

%%% Custom sectioning (sectsty package)
%%% ------------------------------------------------------------
\usepackage{sectsty}

\sectionfont{%                        % Change font of \section command
    \usefont{OT1}{phv}{b}{n}%        % bch-b-n: CharterBT-Bold font
    \sectionrule{0pt}{0pt}{-5pt}{3pt}}

%%% Custom citation
%%% ------------------------------------------------------------
\usepackage[
%   style=ieee,
  backend=bibtex,
  maxnames=99,
  isbn=false,
  doi=false,
  url=false
]{biblatex}
\addbibresource{publication.bib}

\usepackage{xstring}
\usepackage[normalem]{ulem}  %for \uline

\DeclareNameFormat{given-family}{
  \ifthenelse{\equal{\namepartfamily}{Kikuchi}}
    {\uline{\namepartgiveni\addspace\namepartfamily}}
    {\namepartgiveni\addspace\namepartfamily}
  \ifthenelse{\value{listcount} < \value{liststop}}
    {\addcomma}
    {}
}

\DeclareCiteCommand{\fullcite}
  {\usebibmacro{prenote}}
  {\hspace{-0.3em}\usedriver{}{\thefield{entrytype}}.}
  {\multicitedelim}
  {\usebibmacro{postnote}}

%%% Macros
%%% ------------------------------------------------------------
\newlength{\spacebox}
\settowidth{\spacebox}{8888888888}            % Box to align text
\newcommand{\sepspace}{\vspace*{1em}}        % Vertical space macro

\newcommand{\MyName}[1]{ % Name
        \begin{center}
        \Huge \usefont{OT1}{phv}{b}{n} #1
        %\par \normalsize \normalfont
      \end{center}}

\newcommand{\MySlogan}[1]{ % Slogan (optional)
        \large \usefont{OT1}{phv}{m}{n}\hfill \textit{#1}
        \par \normalsize \normalfont}

\newcommand{\NewPart}[1]{\section*{\uppercase{#1}}}

\newcommand{\PersonalEntry}[2]{
        \noindent\hangindent=2em\hangafter=0 % Indentation
        \parbox{\spacebox}{        % Box to align text
        \textit{#1}}               % Entry name (birth, address, etc.)
        \hspace{1.5em} #2 \par}    % Entry value

\newcommand{\SkillsEntry}[2]{      % Same as \PersonalEntry
        \noindent\hangindent=2em\hangafter=0 % Indentation
        \parbox{\spacebox}{        % Box to align text
        \textit{#1}}               % Entry name (birth, address, etc.)
        \hspace{1.5em} #2 \par}    % Entry value

\newcommand{\EducationEntry}[3]{
        \noindent \textbf{#1} \hfill      % Study
        % \colorbox{Black}{%
        %     \parbox{7em}{%
        %     \hfill\color{White}#2}}
      #2 \par  % Duration
        % \noindent \textit{#3} \par        % School
        \begin{addmargin}[2em]{0em}
            #3
        \end{addmargin}}

\newcommand{\WorkEntry}[2]{
        \noindent \textbf{#1} % Project name
        \par
  \noindent\hangindent=1em\hangafter=0\small #2 % Description
        \normalsize \par}

\renewcommand{\labelitemi}{$\triangleright$}
\newcommand{\etal}{\textit{et al}.}
\newcommand{\ie}{\textit{i}.\textit{e}.}
\newcommand{\eg}{\textit{e}.\textit{g}.}

%%% Begin Document
%%% ------------------------------------------------------------
\begin{document}
% you can upload a photo and include it here...
%\begin{wrapfigure}{l}{0.5\textwidth}
%    \vspace*{-2em}
%        \includegraphics[width=0.15\textwidth]{photo}
%\end{wrapfigure}

\MyName{Kotaro Kikuchi}
% \MySlogan{Curriculum Vitae}

% \sepspace

%%% Personal details
%%% ------------------------------------------------------------
\NewPart{Personal details}{}

% \PersonalEntry{Birth}{January 1, 1980}
% \PersonalEntry{Address}{111 First St, New York}
% \PersonalEntry{Phone}{+81 3 3203 4468}
\PersonalEntry{Mail}{kiku-koh@ruri.waseda.jp}
\PersonalEntry{Website}{\url{https://ktrk115.github.io/}}

%%% Education
%%% ------------------------------------------------------------
\NewPart{Education}{}
\begin{itemize}[leftmargin=2em,topsep=0.5em,itemsep=0em]
    \item[] Ph.D. in Computer Science, Waseda University \hfill 2018.4 - 2022.3 (expected)
    \item[] M.Sc. in Computer Science, Waseda University \hfill 2016.4 - 2018.3
    \item[] B.Sc. in Computer Science, Waseda University \hfill 2012.4 - 2016.3
\end{itemize}

%%% Work experience
%%% ------------------------------------------------------------
\NewPart{Research projects}{}
\noindent \textbf{Computational Methods for Graphic Design Applications} \hfill 2018 - Present
\begin{itemize}[leftmargin=2em,topsep=0.5em,itemsep=0em]
    \item \textbf{Latent space exploration for constrained layout generation} \\
    We formalized the problem of controlling the output of a layout GAN
    according to user's requirements as a constrained optimization problem on
    the latent variables of a pre-trained GAN.
    \item \textbf{Hierarchical layout optimization {\small with containment-aware parameterization}} \\
    We proposed an optimization-based method for generating plausible layouts
    for designs with visual containment between elements, such as web pages. Our
    method consists of tree structure estimation and layout generation based on
    the tree structure.
    \item \textbf{Generative adversarial training for object placement} \\
    Spatially placing an object on a background is a basic operation in graphic
    design.  We designed a new regularization method and successfully stabilized
    training of a GAN-based model that predicts the spatial parameters of a
    foreground object.
\end{itemize}

\sepspace

\noindent \textbf{Computer Vision $\times$ Natural Language Processing} \hfill 2015 - 2018
\begin{itemize}[leftmargin=2em,topsep=0.5em,itemsep=0em]
    \item \textbf{Zero-shot image classification by exploiting dictionary definitions} \\
    We enriched semantic representation of object classes by using dictionary
    definitions to improve the accuracy in the zero-shot image classification task.
    (Master's thesis)
    \item \textbf{Video search by textual queries (TRECVID Ad-hoc Video Search)} \\
    We developed a system that uses tens of thousands of concept detectors to
    find videos that involve concepts specified by a textual query. Our system
    won the competition in 2016 and 2017. (Bachelor's thesis)
\end{itemize}

%%% ------------------------------------------------------------
\NewPart{Research Internships}{}

\EducationEntry{CyberAgent AI Lab}{2018.6 - 2018.8}{To generate variations of
banner ads, we built a system that automatically replaces a layer containing a
person in structured visual data such as SVG or PSD, while preserving its
semantics and pose.}

\sepspace

\EducationEntry{Microsoft Research Asia}{2018.9 - 2018.12}{We analyzed latest
video retrieval models and tried to improve their accuracy by incorporating new
attention mechanisms.}

%%% ------------------------------------------------------------
\NewPart{Honors \& Awards}{}
\begin{itemize}[leftmargin=2em,topsep=0.5em,itemsep=0em]
    \item 1st place for manually-assisted setting and 2nd place for
    fully-automatic setting in the TRECVID Ad-hoc Video Search 2017 competition
    \item Student honorable mention in the Meeting on Image Recognition and
    Understanding 2017
    \item 1st place for manually-assisted setting in the TRECVID Ad-hoc Video
    Search 2016 competition
\end{itemize}

%%% ------------------------------------------------------------
\NewPart{Grants}{}
\begin{itemize}[leftmargin=2em,topsep=0.5em,itemsep=0em]
    \item \href{http://www.leading-sn.waseda.ac.jp/en/student/scholarships/}{Scholarship of Graduate Program for Embodiment Informatics} \hfill 2016.4 - 2020.3
\end{itemize}

%%% ------------------------------------------------------------
\NewPart{Research Activities}{}
\begin{itemize}[leftmargin=2em,topsep=0.5em,itemsep=0em]
    \item Student volunteer at International Conference on Computational Linguistics \hfill 2016
\end{itemize}

%%% Skills
%%% ------------------------------------------------------------
\NewPart{Skills}{}
\begin{itemize}[leftmargin=2em,topsep=0.5em,itemsep=0em]
    \item \textbf{Languages}\hspace{1em} Japanese (native), English (upper-intermediate, TOEIC 800, 2016.11)
    \item \textbf{Programming}\hspace{1em} Python (preferred), Google Cloud Platform (e.g. Cloud Dataflow), HTML, CSS, JavaScript, MySQL
\end{itemize}

%%% ------------------------------------------------------------
\NewPart{Publications}{}

\subsection*{Peer-reviewed Conference/Workshop Papers}
\begin{enumerate}[leftmargin=2em,topsep=0.5em,itemsep=0em]
    \item \fullcite{Kikuchi2021}
    \item \fullcite{Kikuchi2019}
    \item \fullcite{Hirakawa2018}
    \item \fullcite{Ueki2018}
    \item \fullcite{Nakatsuka2018}
    \item \fullcite{Kikuchi2017}
    \item \fullcite{Kikuchi2016}
\end{enumerate}

% \setcounter{enumi}{4}

\end{document}