\documentclass[a4paper,11pt]{ltjsarticle}
\usepackage{luatexja} % ltjclasses, ltjsclasses を使うときはこの行不要

\usepackage{hyperref}
\usepackage{scrextend}
\usepackage[english]{babel}
% \usepackage[utf8x]{inputenc}
\usepackage[protrusion=true,expansion=true]{microtype}
\usepackage{amsmath,amsfonts,amsthm}     % Math packages
\usepackage{graphicx}                    % Enable pdflatex
\usepackage[svgnames]{xcolor}            % Colors by their 'svgnames'
\usepackage[top=2cm,bottom=2cm,left=2cm,right=2cm]{geometry}
\usepackage{url}
\usepackage{enumitem}

\frenchspacing              % Better looking spacings after periods
\pagestyle{empty}           % No pagenumbers/headers/footers

%%% Custom sectioning (sectsty package)
%%% ------------------------------------------------------------
% \usepackage{sectsty}

% \sectionfont{%                        % Change font of \section command
%     \usefont{OT1}{phv}{b}{n}%        % bch-b-n: CharterBT-Bold font
%     \sectionrule{0pt}{0pt}{-5pt}{3pt}}

\newcommand{\sectrule}{
    \vspace*{-1.1\baselineskip}
    \noindent\rule{\linewidth}{3pt}}

%%% Custom citation
%%% ------------------------------------------------------------
\usepackage[
%   style=ieee,
  backend=bibtex,
  maxnames=99,
  isbn=false,
  doi=false,
  url=false
]{biblatex}
\addbibresource{publication.bib}

\usepackage{xstring}
\usepackage[normalem]{ulem}  %for \uline

\DeclareNameFormat{given-family}{
  \ifthenelse{\equal{\namepartfamily}{Kikuchi}}
    {\uline{\namepartgiveni\addspace\namepartfamily}}
    {\namepartgiveni\addspace\namepartfamily}
  \ifthenelse{\value{listcount} < \value{liststop}}
    {\addcomma}
    {}
}

\DeclareCiteCommand{\fullcite}
  {\usebibmacro{prenote}}
  {\hspace{-0.3em}\usedriver{}{\thefield{entrytype}}.}
  {\multicitedelim}
  {\usebibmacro{postnote}}

%%% Macros
%%% ------------------------------------------------------------
\newlength{\spacebox}
\settowidth{\spacebox}{888888888888} % Box to align text
\newcommand{\sepspace}{\vspace*{1em}}        % Vertical space macro

\newcommand{\MyName}[1]{ % Name
  \begin{center}
    {\huge #1}
  \end{center}
}

\newcommand{\MySlogan}[1]{ % Slogan (optional)
        \large \usefont{OT1}{phv}{m}{n}\hfill \textit{#1}
        \par \normalsize \normalfont}

\newcommand{\NewPart}[1]{\section*{\uppercase{#1}}\sectrule}

\newcommand{\PersonalEntry}[2]{
        \noindent\hangindent=2em\hangafter=0 % Indentation
        \parbox{\spacebox}{        % Box to align text
        \textit{#1}}               % Entry name (birth, address, etc.)
        \hspace{1.5em} #2 \par}    % Entry value

\newcommand{\SkillsEntry}[2]{      % Same as \PersonalEntry
        \noindent\hangindent=2em\hangafter=0 % Indentation
        \parbox{\spacebox}{        % Box to align text
        \textit{#1}}               % Entry name (birth, address, etc.)
        \hspace{1.5em} #2 \par}    % Entry value

\newcommand{\EducationEntry}[3]{
        \noindent \textbf{#1} \hfill      % Study
        % \colorbox{Black}{%
        %     \parbox{7em}{%
        %     \hfill\color{White}#2}}
      #2 \par  % Duration
        % \noindent \textit{#3} \par        % School
        \begin{addmargin}[2em]{0em}
            #3
        \end{addmargin}}

\newcommand{\WorkEntry}[2]{
        \noindent \textbf{#1} % Project name
        \par
  \noindent\hangindent=1em\hangafter=0\small #2 % Description
        \normalsize \par}

\renewcommand{\labelitemi}{$\triangleright$}
\newcommand{\etal}{\textit{et al}.}
\newcommand{\ie}{\textit{i}.\textit{e}.}
\newcommand{\eg}{\textit{e}.\textit{g}.}

%%% Begin Document
%%% ------------------------------------------------------------
\begin{document}
% you can upload a photo and include it here...
%\begin{wrapfigure}{l}{0.5\textwidth}
%    \vspace*{-2em}
%        \includegraphics[width=0.15\textwidth]{photo}
%\end{wrapfigure}

\MyName{菊池 康太郎}
% \MySlogan{Curriculum Vitae}

% \sepspace

%%% Personal details
%%% ------------------------------------------------------------
\NewPart{連絡先}{}

% \PersonalEntry{Birth}{January 1, 1980}
% \PersonalEntry{Address}{111 First St, New York}
% \PersonalEntry{Phone}{+81 3 3203 4468}
\PersonalEntry{メール}{kiku-koh@ruri.waseda.jp}
\PersonalEntry{ウェブサイト}{\url{https://ktrk115.github.io/}}

%%% Education
%%% ------------------------------------------------------------
\NewPart{学歴}{}
\begin{itemize}[leftmargin=2em,topsep=0.5em,itemsep=0em]
    \item[] {\small 早稲田大学理工学術院 基幹理工学研究科 情報理工・情報通信専攻 博士後期課程} \hfill 2018.4 - 2022.3 (見込み)
    \item[] {\small 早稲田大学理工学術院 基幹理工学研究科 情報理工・情報通信専攻 修士課程} \hfill 2016.4 - 2018.3
    \item[] {\small 早稲田大学 基幹理工学部} \hfill 2012.4 - 2016.3
\end{itemize}

%%% Work experience
%%% ------------------------------------------------------------
\NewPart{研究プロジェクト}{}

\noindent \textbf{\large グラフィックデザイン応用のための計算技術} \hfill 2018 - 現在
\begin{itemize}[leftmargin=2em,topsep=0.5em,itemsep=0em]
    \item \textbf{制約付きレイアウト生成のための潜在空間の探索} \\
    ユーザーの要求に従って出力が制御できるレイアウトの自動生成技術の開発に取り組んだ。
    「要素Aは要素Bの右側に配置」などの要求を制約条件として表現し、
    モデルが生成可能なレイアウトから制約条件を満たすものを効率的に探索する方法を提案した。
    \item \textbf{視覚的内包関係を考慮した階層的レイアウト最適化} \\
    参照デザインに基づいてWebページのレイアウトを自動生成する技術の開発に取り組んだ。
    要素の内包関係を表した木構造を推定した後、その木構造を利用してレイアウトを生成する手法を提案した。
    \item \textbf{オブジェクト配置のための敵対的生成ネットワークの学習安定化} \\
    オブジェクトを画像上の自然な位置に自動的に配置する技術の開発に取り組んだ。
    配置パラメータを予測する敵対的生成ネットワークの学習は不安定であったが、
    新たに提案した正則化手法によって安定した学習が可能になった。
\end{itemize}

\sepspace

\noindent \textbf{\large 画像認識と自然言語処理の融合領域に関する技術} \hfill 2015 - 2018
\begin{itemize}[leftmargin=2em,topsep=0.5em,itemsep=0em]
    \item \textbf{辞書定義文を利用したゼロショット画像認識} \\
    ゼロショット画像認識は、目的クラスの画像情報を学習せずに認識することを目指す試みである。
    画像情報を補完するために辞書の定義文を新たに用いること、またその用い方を提案し、
    公開ベンチマークの認識精度を向上させた。(修士論文)
    \item \textbf{テキストクエリによる映像検索 (TRECVID Ad-hoc Video Search)} \\
    数万個の概念検出器を使って、テキストクエリで指定された概念を含む映像を大規模データセットから
    検索するシステムを開発した。我々のシステムは、2016年と2017年のTRECVIDのコンペティションで
    良い結果を得ました。(学士論文)
\end{itemize}

% \newpage
\clearpage

%%% ------------------------------------------------------------
\NewPart{研究インターンシップ}{}

\EducationEntry{CyberAgent AI Lab}{2018.6 - 2018.8}
{バナー広告のバリエーションを生成するために、SVGやPSDなどの構造的な視覚データの中にある人物を含んだレイヤーを、
その意味合いやポーズを保持したまま、別の人物に自動的に置き換えるシステムを開発しました。}

\sepspace

\EducationEntry{Microsoft Research Asia}{2018.9 - 2018.12}
{最新の映像検索モデルを実装・分析し、さらなる性能改善のために新たな注意機構を取り入れることを検討しました。}

%%% ------------------------------------------------------------
\NewPart{受賞歴}{}
\begin{itemize}[leftmargin=2em,topsep=0.5em,itemsep=0em]
    \item TRECVID Ad-hoc Video Search コンペティション Manually-Assisted System 部門 1位 \hfill 2017 \\
    Fully-Automatic System 部門 2位
    \item 画像の認識・理解シンポジウム (MIRU) 学生奨励賞 \hfill 2017
    \item TRECVID Ad-hoc Video Search コンペティション Manually-Assisted System 部門 1位 \hfill 2016
\end{itemize}

%%% ------------------------------------------------------------
\NewPart{研究助成}{}
\begin{itemize}[leftmargin=2em,topsep=0.5em,itemsep=0em]
    \item \href{http://www.leading-sn.waseda.ac.jp/student/scholarships/}{早稲田大学 博士課程教育リーディングプログラム 実体情報学博士プログラム 奨励金} \hfill 2016.4 - 2020.3
\end{itemize}

%%% ------------------------------------------------------------
\NewPart{その他の研究活動}{}
\begin{itemize}[leftmargin=2em,topsep=0.5em,itemsep=0em]
    \item International Conference on Computational Linguistics 学生ボランティア \hfill 2016
\end{itemize}

%%% Skills
%%% ------------------------------------------------------------
\NewPart{スキル}{}
\begin{itemize}[leftmargin=2em,topsep=0.5em,itemsep=0em]
    \item \textbf{言語}\hspace{1em} 日本語, 英語 (中上級レベル、TOEIC 800, 2016.11)
    \item \textbf{プログラミングなど}\hspace{1em} Python (日常的に使用), Google Cloud Platform (e.g. Cloud Dataflow), HTML, CSS, JavaScript, MySQL
\end{itemize}

\newpage

%%% ------------------------------------------------------------
\NewPart{発表論文}{}
\vspace*{-\baselineskip}
\subsection*{論文誌}
\begin{enumerate}[leftmargin=2em,topsep=0.5em,itemsep=0em]
    \item \fullcite{Kikuchi2021_PG}
\end{enumerate}

\subsection*{国際会議・ワークショップ(査読あり)}
\begin{enumerate}[leftmargin=2em,topsep=0.5em,itemsep=0em]
    \item \fullcite{Kikuchi2021}
    \item \fullcite{Kikuchi2019}
    \item \fullcite{Hirakawa2018}
    \item \fullcite{Ueki2018}
    \item \fullcite{Nakatsuka2018}
    \item \fullcite{Kikuchi2017}
    \item \fullcite{Kikuchi2016}
\end{enumerate}

\subsection*{ワークショップ(査読なし)}
\begin{enumerate}[leftmargin=2em,topsep=0.5em,itemsep=0em]
    \item \fullcite{Ueki2018_trecvid}
    \item \fullcite{Ueki2017_trecvid}
    \item \fullcite{Kikuchi2016_trecvid}
    \item \fullcite{Ueki2016_trecvid}
\end{enumerate}

\end{document}